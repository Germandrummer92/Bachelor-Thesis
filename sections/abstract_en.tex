%% LaTeX2e class for student theses
%% sections/abstract_en.tex
%% 
%% Karlsruhe Institute of Technology
%% Institute for Program Structures and Data Organization
%% Chair for Software Design and Quality (SDQ)
%%
%% Dr.-Ing. Erik Burger
%% burger@kit.edu
%%
%% Version 1.1, 2014-11-21

\Abstract
Cyber-Physical Systems or Hybrid Systems are increasingly important in today's technological society. Different approaches to analyzing and verifying both hybrid models in the form of hybrid programs or - automata are common practice, as well as approaches to analyzing and verifying discrete Java programs. But a  way of gaining a concrete implementation from a hybrid model and verifying the complete system including both the hybrid model and implementation does not exist yet.
In this thesis we close that gap between the two worlds. We propose and apply a way to refine hybrid systems in the form of hybrid programs into concrete Java implementations and offer a formalized approach to verification of the entire systems using a third proof component (aside from the verification of the hybrid model using \keym~and the verification of the Java control program using \key, that we call the glue. 

In the thesis we first showcase a motivating example of the control of a Watertank's water level, detailing the complex issue that is presented by hybrid system refinement and the issues we had with this comparatively easy example. Afterwards we present our approach to refinement of hybrid systems into implementations and verification of hybrid model/implementation packages in a formalized outline of steps to gain a verified implementation. At last we apply our newly found approach to a fresh example of a car - and speed limit controller on a linear track progession ensuring the compliance of the car with the speed limit at all times.