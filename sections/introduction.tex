%% LaTeX2e class for student theses
%% sections/content.tex
%% 
%% Karlsruhe Institute of Technology
%% Institute for Program Structures and Data Organization
%% Chair for Software Design and Quality (SDQ)
%%
%% Dr.-Ing. Erik Burger
%% burger@kit.edu
%%
%% Version 1.1, 2014-11-21

\chapter{Introduction}
\label{ch:Introduction}

The following Bachelorthesis will try to formalize the following process: Replacing the abstract notion of the control program in a verified (by Ke\kern-.2emYmaera) Cyber-Physical-System (\textit{CPS}) with an actual verified (by Ke\kern-.2emY) implementation through a form of Formal Refinement and being able to verify that the entire CPS still satisfys the required safety constraints, using both Ke\kern-.2emYmaera and Ke\kern-.2emY. 

CPS are generally modelled as either Hybrid Automata or - Programs.(See ref.~\cite{platzerb}). Mostly, this means, that an abstract version of the discrete control program is modelled, as a non-deterministic assignment of a control value (See app.~\ref{fig:ex_control}). To replace this non-deterministic assignment with an actual implementation a certain ''glue'' or ''coupling'' has to be found to translate discrete and real continous values into each other.

 In logic this can be expressed as:
\begin{equation}
\centering
\begin{split}
\textbf{If}~ (\models [ controlValue := \ast, ?\psi(controlValue) \dots] \alpha~verified \\ \wedge \textit{``glue''(discreteVariables, continuousVariables)} \\
\textbf{Then}~ [controlValue := JavaProgram \dots] \alpha~also~verified
\end{split}
\label{eq:Main_LogicRefinement}
\end{equation} To explain this process we will take a look at the following: 

\begin{enumerate}[label=\bfseries \Roman*:]

\item Example-based Refinement on CPS Watertank (Example taken from Ke\kern-.2emYmaera).
\item Introduction of Formalized process to gain an actual implementation from a hybrid model.
\item Application of formalized process on example: CPS Gear-Backlash (See ref.~\cite{bla}).
\end{enumerate}

\section{Introducing Cyber-Physical-Systems}
In this thesis we take a close look at \textbf{CPS}. These are systems in which a physical aspect or value is controlled by a computer (program). For example, an aircraft control system in which the computer exerts a form of speed control on the airplane would be a CPS. 

In our case we take a closer look at the closely related notion of \textit{Hybrid Systems}, in which discrete values (in the control program) and continuous values (in the physical world) coexist. The difficulty in analyzing these kinds of systems stems from the ``hybridness'' of the systems: There is always some form of translation necessary to go from the program (discrete values) to the physical (continous reals) world. 

 two basic modelling approaches exist for hybrid systems: Hybrid Automata that are based on Non-Determnistic Finite Automata.





%% -------------------
%% | Example content |
%% -------------------
\iffalse
\todo{This is the SDQ thesis template.
For more information on the formatting of theses at SDQ, please refer to
\url{https://sdqweb.ipd.kit.edu/wiki/Ausarbeitungshinweise} or to your advisor.

\section{Example: Citation}
\label{sec:Introduction:Citation}
A citation: \cite{becker2008a}

\section{Example: Figures}
\label{sec:Introduction:Figures}
\begin{figure}[h]
\centering
\includegraphics[width=4cm]{logos/sdqlogo}
\caption{SDQ logo}
\label{fig:sdqlogo}
\end{figure}

The SDQ logo is displayed in \autoref{fig:sdqlogo}.

\section{Example: Tables}
\label{sec:Introduction:Tables}
\begin{table}[h]
\centering
\begin{tabular}{r l}
\toprule
abc & def\\
ghi & jkl\\
\midrule
123 & 456\\
789 & 0AB\\
\bottomrule
\end{tabular}
\caption{A table}
\label{tab:atable}
\end{table}

\section{Example: Todo-Note}
Meaningless text.
Replace with meaningful text. (This note is only shown in draft mode.)

\section{Example: Formula}
One of the nice things about the Linux Libertine font is that it comes with
a math mode package.
\begin{displaymath}
f(x)=\Omega(g(x))\ (x\rightarrow\infty)\;\Leftrightarrow\;
\limsup_{x \to \infty} \left|\frac{f(x)}{g(x)}\right|> 0
\end{displaymath}
}
\fi
%% --------------------
%% | /Example content |
%% --------------------