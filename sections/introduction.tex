%% LaTeX2e class for student theses
%% sections/content.tex
%% 
%% Karlsruhe Institute of Technology
%% Institute for Program Structures and Data Organization
%% Chair for Software Design and Quality (SDQ)
%%
%% Dr.-Ing. Erik Burger
%% burger@kit.edu
%%
%% Version 1.1, 2014-11-21

\chapter{Introduction}
\label{ch:Introduction}

A growing number of automation today is done through computers and computer programs that reach outside of their digital world into the analog world, controlling some physical aspects. These systems are called cyber-physical systems (\textit{CPS}) and are always safety-critical, due to the physical control they exert. While formal verification methods exist for hybrid models of CPS and formal verification methods exist for concrete implementations, a certain gap is evident in between the two, when trying to implement control programs for concrete CPSs.

In this bachelorthesis we formalize an approach attempting to close that gap: Replacing the abstract notion of the control program in a verified (by \keym) Cyber-Physical-System (\textit{CPS}) with a concrete, verified (by \key) implementation through a form of Formal Refinement and being able to verify that the entire CPS still satisfies the required safety constraints, using \keym. 

CPS are generally modelled as either Hybrid Automata or - Programs.(See ref.~\cite{platzer2010b}). In these hybrid models, the control-part only exists in an implicit manner, while we require an explicit statement at which the control program is enacted, what we later on refer to as the ``hook''.  

To be able to verifiy the entire implementation against the cps-safety-constraint, we define a ``glue'', which refers to a relation that translates real values into discrete values and vice-versa. In certain cases (See sect.~\ref{sec:Watertank:Glue}), glue will be a concrete function and not only a relation, but not in general.

If we summarize the goal of this thesis we get ~\ref{eq:Main_LogicRefinement}. Here, glue is the aforementioned relation to be able to translate from the real into the discrete world.


\begin{table}[]
\centering
\caption{This thesis' main goal}\label{eq:Main_LogicRefinement}
\begin{tabular}{l}
\begin{tabular}[c]{@{}l@{}}1. Verification of hybrid model against safety property \(\alpha\) using \keym. \\
2. Verification of implementation using \key. \\
3. Verification of glue using \keym.
\end{tabular} \\ \hline
\(\implies\) Hybrid model with concrete implementation inserted at hook fullfills safety property \(\alpha\).
\end{tabular}
\end{table}

This means: if we have a certain hybrid program with an explicit hook that models a CPS and fullfills safety condition \(\alpha\) (meaning the CPS always terminates and is always in a state which fullfills \(\alpha\)), an implementation that is correct and have found a valid, verified glue relation between the different values of the physical and discrete system-parts, then the hybrid program using the concrete implementation in place of the hook also fullfills safety condition \(\alpha\).


\section{Related Work}
\label{sec:related}

\section{Outline}
\label{sec:Outline}

In this thesis we take a look at the following:

\begin{enumerate}[label=\bfseries \Roman*:]

\item Preliminary Definitions.
\item Motivating Example: Refining a concrete case study: CPS ``Watertank'' (Example taken from~\cite{keymaeraGuide}).
\item Introduction of approach to using Refinement to gain a concrete implementation from a hybrid model.
\item Application of newly found approach: another case study: CPS ``Traffic Control'' (See ref.~\cite{bla}).
\end{enumerate}