%% LaTeX2e class for student theses
%% sections/abstract_de.tex
%% 
%% Karlsruhe Institute of Technology
%% Institute for Program Structures and Data Organization
%% Chair for Software Design and Quality (SDQ)
%%
%% Dr.-Ing. Erik Burger
%% burger@kit.edu
%%
%% Version 1.1, 2014-11-21

\Abstract
Cyber-Physical Systeme, für uns äquivalent zu Hybriden Systemen, werden heutzutage immer wichtiger. Verschiedene Ansätze um hybride Modelle in der Form von hybriden Programmen, als auch -automaten zu analysieren und verifizieren sind heute schon sehr üblich. Für Ansätze die diskrete Java Programme analysieren und verifizieren gilt dies genauso. Aber einen konkreten Ansatz um aus einem hybriden Modell eine konkrete Implementation zu erhalten existiert noch nicht. In dieser Arbeit schließen wir diese Lücke. Wir stellen einen Ansatz vor mit dem man Hybride Systeme gegeben in der Form von Hybriden Programmen in konkrete Java Implementationen refinen kann und wenden diesen an. Außerdem stellen wir einen formalisierten Ansatz vor um das komplette System bestehend aus dem Hybriden Modell und dem Java Kontrollprogramm zu verifizieren. Dafür führen wir einen dritten Beweisbaustein ein (Neben der Verifizierung des Hybriden Modelles und des Java Programmes mit \keym~ und \key~resp.) den wir als  ``Glue'' bezeichnen.

In dieser Arbeit stellen wir daher erst einmal ein motivierendes Beispiel in der Form eines Wassertanks, dessen Wasserlevel durch Anpassung des Ventils der Zuflussgeschwindigkeit in einem bestimmten Bereich gehalten werden soll vor. Das Beispiel zeigt die Komplexität die das Refinement eines Hybriden Systems mit sich bringt, selbst bei einem so einfachen Beispiel wie dem Wassertank. Danach stellen wir unseren Ansatz, wie man aus einem hybriden modells eine konkrete Implementation gewinnt und diese zusammen mit dem hybriden Modell verifiziert, konkret vor. Als letzes wenden wir unseren vorgestellten Ansatz dann auf ein frisches Beispiel an. In diesem wird die Beschleunigung eines Autos auf einer linearen Strecke kontrolliert und Geschwindigkeitslimits von einem weiteren Kontrollprogramm auf dieser Strecke verteilt. Das Auto und die Geschwindigkeitslimits sollen dann so kontrolliert werden, dass das Auto die Geschwindigkeitslimits zu jeder Zeit einhält.
