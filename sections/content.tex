%% LaTeX2e class for student theses
%% sections/content.tex
%% 
%% Karlsruhe Institute of Technology
%% Institute for Program Structures and Data Organization
%% Chair for Software Design and Quality (SDQ)
%%
%% Dr.-Ing. Erik Burger
%% burger@kit.edu
%%
%% Version 1.1, 2014-11-21

%To be able to reference labels in other document
\externaldocument{introduction}

%%%%%%%%%%%%%%%%%%%%%%%%%%%%%%%%%%%%%%%%%%%%%%%%%%%%%%%%%%%%%%%%%%%%%
%--WATERTANK--%
%%%%%%%%%%%%%%%%%%%%%%%%%%%%%%%%%%%%%%%%%%%%%%%%%%%%%%%%%%%%%%%%%%%%%
\chapter{Example-based Refinement on CPS Watertank}
\label{ch:Watertank}

To get a better understanding of the tasks involved in refining a hybrid model into a implementation with all necessary intermediate verification steps, we used one of the out-of-the-box examples provided in the \keym~tutorial~\cite{keYmaera}.

\section{Finding the concrete Control Value Assisgnment}
\label{sec:Watertank:ControlValue}

In order to be able to apply Eq.~\ref{eq:Main_LogicRefinement} to this concrete example, the first challenge we faced was finding a spot in which a concrete control value is actually assigned. Taking a look at the Hybrid Automata describing the Watertank (See Fig.~\ref{fig:HA_Watertank}), 
\section{Refining the original Hybrid Automata}
\label{sec:Watertank:Refinining}

\section{Finding the correct Program Safety Condition}
\label{sec:Watertank:SafetyCond}

\section{The (simple) Java Control Program}
\label{sec:Watertank:Java}

\section{Finding the glue between Java and the Hybrid Model of the system}
\label{sec:Watertank:Glue}

\section{Verification based on \keym}
\label{sec:Watertank:Verification}

%%%%%%%%%%%%%%%%%%%%%%%%%%%%%%%%%%%%%%%%%%%%%%%%%%%%%%%%%%%%%%%%%%%%%
%--PROCESS--%
%%%%%%%%%%%%%%%%%%%%%%%%%%%%%%%%%%%%%%%%%%%%%%%%%%%%%%%%%%%%%%%%%%%%%
\chapter{Introduction of formalized process of using Refinement to gain a concrete implementation from a hybrid model}
\label{ch:Process}

What the Watertank example shows is the non-triviality of refining the hybrid model into an implementation and of the verification of all necessary parts. Overall it is obvious, that a formalized approach to the general problem presented in chapter~\ref{ch:Introduction} is necessary. In this chapter we present a possible formalized approach to the problem, that we deem feasible.
\\

To aid readability we will now give an overview of the process without explanation, then detailing each step in the following sections. 

\begin{enumerate}
\item Abstraction of the original hybrid model to better split actual control system ``hook'' and physical evolutions.
\item Finding the necessary safety condition of the control value for verification with \keym.
\item Implementing control program according to safety condition as its specification and Verification by \key.
\item Finding the correct ``glue'' between hybrid model and control program and its verification by \keym.
\item Result evaluation: Was eq. ~\ref{eq':Main_LogicRefinement} proven?
\end{enumerate}

\section{Abstracting original hybrid model to better split actual control system ``hook'' and physical evolutions}
\label{sec:Process:Hook}
Most CPS we took a look at (See \cite{keymaera} Tutorial, \cite[p.~5, p.~11]{platzer2010b} \dots) as examples, did not have a concrete spot in which a control program could ``hook'' in easily. This means, that the first step in our refinement process has to be finding a suitable hook for the control program, referring to one or more non-deterministic assignments of a/multiple control values.

\section{Finding the necessary safety condition of the control value for verification with \keym}
\label{sec:Process:SafetyCond}

\section{Implementing control program according to safety condition as its specification and Verification by \key.}
\label{sec:Process:Implementation}

\section{Finding the correct glue between hybrid model and control program and verifying it with \keym}
\label{sec:Process:Glue}

\section{Evaluating results}
\label{sec:Process:Eval}










%% -------------------
%% | Example content |
%% -------------------
\iffalse
The content chapters of your thesis should of course be renamed. How many
chapters you need to write depends on your thesis and cannot be said in general.

Check out the examples theses in the SDQWiki:

\url{https://sdqweb.ipd.kit.edu/wiki/Abschlussarbeit/Studienarbeit}

Of course, you can split this .tex file into several files if you prefer. 


\section{First Section}
\label{sec:FirstContent:FirstSection}

\dots

\section{Second Section}
\label{sec:FirstContent:SecondSection}

\dots


\chapter{Second Content Chapter}
\label{ch:SecondContent}

\dots

\section{First Section}
\label{sec:SecondContent:FirstSection}

\dots

\section{Second Section}
\label{sec:SecondContent:SecondSection}

\dots

Add additional content chapters if required by adding new .tex files in the
\code{sections/} directory and adding an appropriate 
\code{\textbackslash include} statement in \code{thesis.tex}. 
\fi
%% ---------------------
%% | / Example content |
%% ---------------------